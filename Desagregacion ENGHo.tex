

\documentclass[11pt,spanish]{article}
\usepackage[T1]{fontenc}
\usepackage{selinput}
\SelectInputMappings{%
  aacute={á},
  ntilde={ñ},
}
%%\usepackage[latin9]{inputenc}

\usepackage{array}
\usepackage{float}
\usepackage{amsmath}
\usepackage{graphicx} 

\title{Desagregar ENGHo} 

\date{\vspace{-5ex}}

\begin{document}
\maketitle
\section{Introducción}
El domino de estimación de la ENGHo  abarca a las personas residentes en los hogares de viviendas particulares de las localidades de la República Argentina con 2.000 o más habitantes. El domino de estimación geográfico mas pequeño es la provincia (y CABA).\\
Por otro lado, la EPH releva información sobre los aglomerados urbanos con mas de 100 mil habitantes y las capitales de provincia.\\
La principal diferencia en lo que respecta a cobertura entre ambas encuestas es que la ENGHo es mas amplia en términos geográficos ya que considera todas las localidades con mas de 2 mil habitantes. El universo bajo estudio de la ENGHo contiene al de la EPH (esta es la clave!).\\
Seria interesante poder diferenciar en la ENGHo, para ciertas variables (empleo, pobreza, etc), sus valores para los grandes aglomerados urbanos (cubiertos por la EPH) y para las pequeñas localidades (El Interior). \\
El objetivo del trabajo es poder estimar ciertas variables en forma diferenciada, para los grandes aglomerados urbanos y para el Interior. 

\section{Modelo}
Supongamos que tenemos una variable $y$ y dos zonas diferentes, $A$ y $B$, ademas, supongamos que estamos interesados en estimar la media de $y$ diferenciando por zonas. El modelo viene dado por:\\
\begin{equation}\label{Modelo}
y=\beta_{A}I_{A}+\beta_{B}I_{B}+u
\end{equation}
donde $I_{A}$ e $I_{B}$ son variables indicadoras de las zonas $A$ y $B$ respectivamente.\\
De (\ref{Modelo}) tenemos que la media de $y$ en la zona $A$ viene dada por $\beta_{A}$ y la media de $y$ en la zona $B$ viene dada por $\beta_{B}$.\\
Por ejemplo, supongamos que $y$ es el ingreso, $A$ un aglomerado urbano (Ej: Gran Rosario) y $B$ el Interior (de Santa Fe). No podemos estimar la media del ingreso en cada zona porque no observamos ni $I_{A}$ ni $I_{B}$ ya que la ENGHo no hace esa diferencia.\\
Ahora bien, la media \textbf{no condicional} de $y$ viene dada por,
\begin{equation*}
E(y)=\beta_{A}E(I_{A})+\beta_{B}E(I_{B})+E(u)
\end{equation*}
La $E(I_{A})=p_{a}$ y la $E(I_{B})=p_{b}$, siendo $p_{a}$ y $p_{b}$ la proporción de elementos de la población que pertenecen a la zona $A$ y $B$ respectivamente. Asumiendo que  $E(u)=0$, tenemos que
\begin{equation}\label{Modelo 1}
E(y)=\beta_{A}p_{a}+\beta_{B}p_{b}
\end{equation}

\section{Metodología}
De (\ref{Modelo 1}) tenemos que,
\begin{equation*}
\beta_{B}= \dfrac{E(y)-\beta_{A}p_{a}}{p_{b}}
\end{equation*}
$\beta_{A}$ es la media de $y$ en la zona $A$ (un aglomerado urbano) la cual puede ser estimada con la EPH, $p_{a}$ y $p_{b}$ son las proporciones del total poblacional  que corresponden al aglomerado urbano y al interior, esto se puede obtener porque conozco el universo (total poblacional al que apuntan) de la EPH y de la ENGHo. El universo de la ENGHo contiene al de la EPH. Por ultimo, la $E(y)$ es la media no condicional (para ambas zonas conjuntamente) de $y$ que se puede obtener tomando el promedio de $y$ en la ENGHo. \\
Con lo anterior podríamos estimar $\beta_{B}$ de la siguiente manera,
\begin{equation}\label{estimador}
\tilde{\beta}_{B}= \dfrac{\hat{E}(y)-\hat{\beta}_{A}\hat{p}_{a}}{\hat{p}_{b}}
\end{equation}

Los pasos a seguir vienen dados por:
\begin{itemize}
\item Estimo $\beta_{A}$ de la EPH. Esto es la media de $y$ para un aglomerado urbano.
\item Según el universo  de la ENGHo y la EPH calculo $p_{a}$ y $p_{b}$. Lo que representa de la población el aglomerado y el interior.
\item Estimo $E(y)$ de la ENGHo. Es el promedio de $y$ considerando ambas zonas conjuntamente.
\item Aplico (\ref{estimador})
\end{itemize}

\section{Comentarios}
\begin{itemize}
\item Las variables que se pueden usar son aquellas comunes a ambas bases. Cuales son?? son interesantes??
\item El modelo puede extenderse fácilmente a $k$ zonas. Varios aglomerados vs. El Interior.
\end{itemize}

\end{document}